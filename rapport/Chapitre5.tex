\section*{Chapitre 5}
\section{Développement}
	La partie développement s'est déroulée en trois sous-parties. Tout d'abord, nous nous sommes informés des méthodes proposées sur Android pour mettre en place les fonctionnalités précisés dans le Chapitre 4. Ensuite nous avons ajouté plusieurs catégories de questions/réponses sur le site Pandorabots pour faire fonctionner nos fonctionnalités. Enfin, nous avons modifié l'interface utilisateur de l'application pour améliorer l'expérience d'utilisation. Dans ce cas là, Nous avons changé le personnage d'animation pour pouvoir afficher davantage d'informations sur l'activité principale. Ces trois sous-parties ont été implémentées itérativement tout au long du développement. Nous allons parler de deux sous-parties dernières car les détails et l'algorithme de sous-partie 1 ont été précisés dans le chapitre précedante.

\subsection{Ajout de plusieurs catégories de questions/réponses sur le site Pandorabots}

\indent Nous continuions à utiliser le compte créé par les étudiants précédents pour ajouter des catégories de questions/réponses sur \url{https://www.pandorabots.com/botmaster/fr/home}, avec les identifiants suivants:

\begin{center}
	Email : \textbf{\emph{alexandre.levacher@insa-rouen.fr}}\\
	Mot de passe : \textbf{\emph{PAOCompagnonVirtuel3}}
\end{center}

\indent Tout les informations d'une quesion/réponse sont comprises entre \textbf{\emph{<category>}} et \textbf{\emph{</cagegory>}}. La balise \textbf{\emph{<pattern>}} contient la question et la balise \textbf{\emph{<template>}} est la partie de réponse intelligente. La balise \textbf{\emph{<oob>}} se situe dans la balise \textbf{\emph{<template>}}.\\
\indent S'il y a des étoiles entre la balise \textbf{\emph{<pattern>}}, nous allons ajouter les balises entre la balise \textbf{\emph{<oob>}}, par exemple : \textbf{\emph{<add><star index="1"/></add>}}. La balise \textbf{\emph{<add>}} représente l'action à faire.

\subsubsection{Ajouter un réveil sans répétition}
\begin{lstlisting}[frame=none,aboveskip=0.5em]
<category><pattern>SONNER POUR *</pattern><template><oob><alarmclock>
<add><star/></add></alarmclock></oob></template></category>
\end{lstlisting}

\begin{lstlisting}[frame=none,aboveskip=0.5em]
<category><pattern>AJOUTER UN RÉVEIL À * </pattern><template><srai>SONNER POUR <star/></srai></template></category>
\end{lstlisting}

\begin{lstlisting}[frame=none,aboveskip=0.5em]
<category><pattern>AJOUTER UN RÉVEIL POUR * </pattern><template><srai>SONNER POUR <star/></srai></template></category>
\end{lstlisting}

\subsubsection{Ajouter un réveil avec répétition}
\begin{lstlisting}[frame=none,aboveskip=0.5em]
<category><pattern>AJOUTER UN RÉVEIL À * POUR TOUS LES * </pattern><template><oob><alarmclock><add><star index="1"/></add><repetition><star index="2"/></repetition></alarmclock></oob></template></category>
\end{lstlisting}

\begin{lstlisting}[frame=none,aboveskip=0.5em]
<category><pattern>AJOUTER UN RÉVEIL POUR * POUR TOUS LES * </pattern><template><srai>AJOUTER UN RÉVEIL À<star index="1"/>POUR TOUS LES<star index="2"/></srai></template></category>
\end{lstlisting}

\begin{lstlisting}[frame=none,aboveskip=0.5em]
<category><pattern>AJOUTER UN RÉVEIL POUR TOUS LES * POUR * </pattern><template><srai>AJOUTER UN RÉVEIL À<star index="2"/>POUR TOUS LES<star index="1"/></srai></template></category>
\end{lstlisting}

\begin{lstlisting}[frame=none,aboveskip=0.5em]
<category><pattern>AJOUTER UN RÉVEIL POUR TOUS LES * À * </pattern><template><srai>AJOUTER UN RÉVEIL À<star index="2"/>POUR TOUS LES<star index="1"/></srai></template></category>
\end{lstlisting}

\subsubsection{Supprimer un réveil}
\begin{lstlisting}[frame=none,aboveskip=0.5em]
<category><pattern>SUPPRIMER UN RÉVEIL DE *</pattern><template><oob>
<alarmclock><delete><star/></delete></alarmclock></oob></template></category>
\end{lstlisting}

\begin{lstlisting}[frame=none,aboveskip=0.5em]
<category><pattern>SUPPRIMER UN RÉVEIL À *</pattern><template><srai>SUPPRIMER UN RÉVEIL DE <star/></srai></template></category>
\end{lstlisting}

\subsubsection{Ajouter un rendez-vous}
\begin{lstlisting}[frame=none,aboveskip=0.5em]
<category><pattern>AJOUTER UN RENDEZ-VOUS LE *</pattern><template><oob><appointment><add><eventdate><star/></eventdate></add></appointment></oob></template></category>
\end{lstlisting}

\begin{lstlisting}[frame=none,aboveskip=0.5em]
<category><pattern>AJOUTER UN RENDEZ-VOUS LE * À * , *</pattern><template><oob><appointment><add><eventdate><star index="1"/></eventdate><eventtime><star index="2"/></eventtime><event><star index="3"/></event></add></appointment></oob></template></category>
\end{lstlisting}

\begin{lstlisting}[frame=none,aboveskip=0.5em]
<category><pattern>AJOUTER UN RENDEZ-VOUS LE * À *</pattern><template><oob><appointment><add><eventdate><star index="1"/></eventdate><eventtime><star index="2"/></eventtime></add></appointment></oob></template></category>
\end{lstlisting}

\subsubsection{Supprimer un rendez-vous}
\begin{lstlisting}[frame=none,aboveskip=0.5em]
<category><pattern>SUPPRIMER UN RENDEZ-VOUS LE *</pattern><template><oob><appointment><delete><eventdate><star/></eventdate></delete></appointment></oob></template></category>
\end{lstlisting}

\begin{lstlisting}[frame=none,aboveskip=0.5em]
<category><pattern>SUPPRIMER UN RENDEZ-VOUS LE * À * INTITULÉ *</pattern><template><oob><appointment><delete><eventdate><star index="1"/></eventdate><eventtime><star index="2"/></eventtime><event><star index="3"/></event></delete></appointment></oob></template></category>
\end{lstlisting}

\begin{lstlisting}[frame=none,aboveskip=0.5em]
<category><pattern>SUPPRIMER UN RENDEZ-VOUS LE * INTITULÉ *</pattern><template><oob><appointment><delete><eventdate><star index="1"/></eventdate><eventtime>0</eventtime><event><star index="2"/></event></delete></appointment></oob></template></category>
\end{lstlisting}

\subsubsection{Réaliser une recherche en Google Map}
\begin{lstlisting}[frame=none,aboveskip=0.5em]
<category><pattern>_MAPS *</pattern><template><oob><maps><star/></maps></oob></template></category>
\end{lstlisting}

\begin{lstlisting}[frame=none,aboveskip=0.5em]
<category><pattern>_ MAPS *</pattern><template><oob><maps><star/></maps></oob></template></category>
\end{lstlisting}

\subsubsection{Envoyer un sms}
\begin{lstlisting}[frame=none,aboveskip=0.5em]
<category><pattern>_ENVOYER UN SMS POUR * POUR DIRE *</pattern><template><oob><send><people><star index="1"/></people><message><star index="2"/></message></send></oob></template></category>
\end{lstlisting}

\begin{lstlisting}[frame=none,aboveskip=0.5em,belowskip=1.5em]
<category><pattern>_ENVOIE UN SMS POUR * POUR DIRE *</pattern><template><oob><send><people><star index="1"/></people><message><star index="2"/></message></send></oob></template></category>
\end{lstlisting}



\subsection{Modification de l'interface d'utilisateur}
\indent Premièrement, nous avons séparé le centre de l'interface utilisateur aux deux parties. La partie gauche affiche les cinq reconnaissances vocales les plus probables de la voix de l'utilisateur. La partie droite affiche la réponse de notre application si elle existe. En bas de l'écran, nous avons ajouté un \textbf{\emph{textField}} pour afficher le prochain événement dans le calendrier. Ce \textbf{\emph{textField}} va se mettre à jour s'il existe un changement d'événement plus proche dans le calendrier grâce à la fonction \textbf{\emph{getNextEvent}}. Autrement, nous avons changé le personnage d'animation. Nous allons parler dans la paragraphe suivante.

\subsection{Changement du personnage}

\indent Nous avons choisi un personnage d'animation de Microsoft qui est open source (le lien d'image source est donné dans la partie bibliographie). Nous avons téléchargé l'ensemble d'images du personnage tout d'abord. Puis, nous avons coupé l'image sur un site web. Pour différentes actions, nous avons proposé un temp de continu. Avec ces images, nous avons réalisé les animations  suivantes dans la liste: \\
\indent saluer\\
\indent écouter\\
\indent trouver une  erreur\\
\indent parler\\
\indent rire\\
\indent répondre\\
\indent Ce changement du personnage améliore notre l'interface d'utilisateur, de plus, il nous permet d'avoir plus d'espace sur l'écran pour afficher d'autres informations. Nous pensons que c'est un changement indispensable.

\begin{lstlisting}[frame=none,aboveskip=0.5em]
<animation-list
    android:id="@+id/selected"
    android:oneshot="false"
    xmlns:android="http://schemas.android.com/apk/res/android">
    <item android:drawable="@drawable/greet1" android:duration="200" />
    <item android:drawable="@drawable/greet2" android:duration="200" />
    <item android:drawable="@drawable/greet3" android:duration="200" />
    <item android:drawable="@drawable/greet4" android:duration="200" />
    <item android:drawable="@drawable/greet5" android:duration="200" />
    <item android:drawable="@drawable/greet6" android:duration="200" />
    <item android:drawable="@drawable/greet7" android:duration="200" />
    <item android:drawable="@drawable/greet8" android:duration="800" />
</animation-list>
\end{lstlisting}

\indent C'est le code pour générer l'animation 'saluer'. Chaque ligne représente une image et sa durée d'affichage en unité de ms.
\newpage
