\section*{Conclusion}

\indent Sur le plan technique, la réalisation d'ajout des fonctionnalités pour le compagnon virtuel dans le cadre de notre PAO nous permet d'apprendre les nouvelles connaissances. Dans notre PAO, nous avons appris les connaissances de base du développement sur Android et comment écrire un fichier AIML pour réaliser un chatbot, surtout nous avons découvert un nouveau domaine qui nous intéresse beaucoup. En même temps, nous avons utilisé les connaissances et techniques que nous avons appris en sixième semestre - le développement en JAVA. Nous voudrons détailler plus pour la préparation de PAO. Pour travailler sur ce PAO, nous nous sommes formés avec les cours fondamentaux sur Internet. Nous avons appris comment construire une interface d'utilisateur et ses animations avec les fichiers XML, debugger sous IDE Android Studio, etc. Nous pensons que c'est une très bonne expérience d'apprendre comment se former pour travailler sur un projet.

\indent Sur le plan méthodologique, ce projet nous a permis de mettre en place les méthodes de travail propres à la gestion de projet tel que le cycle en V. Nous avons séparé 6 étapes dans ce PAO. Premièrement, nous avons listé les tâches à faire ou les fonctionnalités à réaliser, c'est l'étape de l'analyse des besoins. Deuxièmement, selon les besoins, nous avons proposé les solutions et donner chaque solution un diagramme correspondant. Troisièmement, nous avons listé les signatures de fonctions ou procédures à développer. Quatrièmement, nous avons écrit les algorithmes pour les fonctions ou procédures compliquées. Ensuite, c'est la partie de développement et test. Nous pensons que nous avons suivi le cycle en V pour le développement. Chaque semaine, nous faisons des réunions avec M. Pauchet pour nous aider si nous avions des problèmes. Ainsi, ce projet nous a permis d'introduire le prochain projet effectué en groupe le prochain semestre, le PIC (Projet INSA Certifié).

\indent Le résultat final est un succès : excepté le problème de l'affichage de Google Map dans notre application, toutes les fonctionnalités et les tâches listées sont intégrées à notre application, et autrement, nous avons réussi à améliorer l'interface utilisateur avec le nouveau personnage d'animation et le nouveau arrangement des compositions. Nous avons une vidéo pour démontrer ce que nous avons réalisé dans les rendus. Il reste quand même beaucoup de travail intéressant à fournir sur ce projet, nous souhaitons que le prochain groupe réussisse à les réaliser. Enfin, nous remercions M. Pauchet pour avoir nous donné ce projet et nous guidé durant son déroulement ce semestre.

\newpage
