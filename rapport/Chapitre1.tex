\section*{Chapitre 1}
\section{Analyse des besoins}
\subsection{Liste des besoins}
Le but principal du projet est de développer des nouvelles fonctionnalités sur l'application Compagnon Virtuel déjà existante, notre version sera la version 4.
Avant de commencer le projet, l'application est déjà fonctionnelle, elle propose une interaction avec un personnage animé. Ce dernier est capable d'échanger via reconnaissance vocale et synthèse vocale. L'analyse des requêtes est déportée sur un serveur externe de dialogue intelligent: pandorabots. Le compagnon virtuel s'exprime aussi via des animations. Pendant la préparation du projet, nous avons proposé de réaliser les fonctionnalités suivantes:\\
	\indent- effectuer une recherche dans googlemaps et afficher une carte dans l'application;\\
	\indent- envoyer des sms;\\
	\indent- ajouter et supprimer un réveil;\\
	\indent- afficher un calendrier au sein de l'application;\\
	\indent- intégrer le processus de gestion de calendrier au chatbot externe;\\
	\indent- afficher le prochain événement dans le calendrier, il est capable de mettre à jour le prochain événement s'il y a une modification dans le calendrier.
	
Autrement, nous avons proposé d'améliorer l'interface utilisateur de l'application car nous n'étions pas satisfaits avec l'interface utilisateur actuelle.

\subsection{Rendus}
En résumé, les rendus demandés sont:\\
    \indent- une application fonctionnelle sous Android,\\
    \indent- un rapport sur le déroulement du projet,\\
    \indent- une documentation du code source,\\
    \indent- un guide utilisateur,\\
	\indent- une vidéo pour faire une démonstration de l'application.\\
\indent Ils doivent être fournis avant la fin du premier semestre de ASI 4.1, afin d'être compabilisés dans la moyenne semestrielle.
\newpage
