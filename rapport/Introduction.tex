\section*{Introduction}


Dans le cadre de notre scolarité au sein du département Architecture des Systèmes d’Information (ASI) de l’Institut National des Sciences Appliquées (INSA) de Rouen, nous devons réaliser au cours de notre premier semestre de quatrième année un Projet d’Approfondissement et d’Ouverture (PAO), c’est-à-dire un travail sur plusieurs mois, seul ou en équipe, dans lequel nous pouvons mettre en pratique nos connaissances apprises en cours et/ou apprendre de nouvelles connaissances sur des sujets en rapport avec notre formation.
Nous avons donc tous les deux décidé de nous intéresser à un sujet proposé par M. Alexandre Pauchet : la création d’un compagnon virtuel sous Android. Ce PAO permet d’une part de mettre en application nos acquis en Java de troisième année, et d’autre part de découvrir la conception d’une application mobile sous Android, chose totalement nouvelle pour nous deux.
