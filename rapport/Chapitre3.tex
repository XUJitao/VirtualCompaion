\section*{Chapitre 3}
\section{Conception Préliminaire}
\indent Cette étape nous permet, à partir des différents éléments de l'analyse, de mettre en forme les fonctions et procédures afin d'en expliciter les nouveaux fonctionnements.

\subsection{Cas d'utilisation}
\begin{figure}[h]
\centering
\includegraphics[scale=0.5]{./diagrammes/UsecaseDiagram.jpeg}
\caption{Cas d'utilisation.\label{fig2}}
\end{figure}

\subsection{Diagrammes de Sequence}
\begin{figure}[h]
\centering
\includegraphics[scale=0.4]{./diagrammes/SequenceDiagram.jpeg}
\caption{Diagramme de Sequence pour envoyer un sms.\label{fig3}}
\end{figure}
\indent C'est un diagramme de sequence pour envoyer un sms. Les parties à la boite noire est la partie du code qui a été complété dans les dernières versions. Nous vous donnons ce diagramme comme un model de diagramme de sequence. Pour autres fonctions ou procédures, nous travaillons en même mode.
\newpage
\newpage


\subsection{Signatures Partie Chatbot}
procédure setAppointement (E/S Gcal : GestionCalendar, E oobContent : String, operationType : String)\\
\indent fonction setBeginTimeAndGetTitle (E oobContent : String, beginTime : Calendar, operationType : String) : String\\
\indent procédure googleQuery (E/S googleSearchText : String)\\
\indent procédure launchApp (E app : String)\\
\indent procédure launchUrl (E/S url : String)\\
\indent procédure launchGoogleMap (E/S address : String)\\
\indent procédure sendSMS (E oobContent : String)\\
\indent procédure makePhoneCall (E oobContent : String)\\
\indent procédure addAnAlarm(E oobContent : String)\\
\indent procédure deleteAnAlarm (E oobContent : String)\\

\subsection{Signatures Partie ToolManager}
procédure setNextEvent()\\
\indent fonction getNextEvent() : String\\
\newpage

